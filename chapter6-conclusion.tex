\chapter{\label{chap:conclusion}Conclusion}

The goal of this thesis was to develop a model that is able to classify sports poses from images and uses self-supervised learning to achieve better results on datasets with small amount of annotated samples. Time-contrastive learning was chosen as the approach to achieve self-supervision on the data. For that, a set of tools for video preparation had to be implemented and tested on various scenes. After collecting and preparing of a dataset of sports poses, a self-supervised model consisting of encoder and classifier was developed. The self-supervised model was evaluated in comparison to model trained with supervision and results were presented and discussed. 

Set of tools for dataset preparation was developed and it is effectively working on any number of videos of a different scenes. The tools are able to trim and crop videos easily with the least amount of manual work. Synchronization tool can automatically adjust the video start times and lengths to align any number of videos. A useful dataset of images for time-contrastive learning is then detected by using sparse optical flow and exported for future use. Simple tool for labeling of images is also implemented.

Two datasets were recorded and prepared with the mentioned tools. The first one contains hand gestures with different backgrounds and was mainly used for testing of the dataset preparation tools. The second one contains recordings of upper body with diverse arm movement. This dataset was used to train and evaluate the self-supervised model. It contains 3804 images of sports poses with two sets of annotations of 4 and 16 classes.

Two models constructing the self-supervision architecture were developed -- encoder and classifier. The encoder uses ResNet-50 architecture together with triplet loss function to provide embeddings of sports poses. The classifier is a simple dense neural network that takes embedding vectors and classifies them. Another model doing the same task but with supervision was also developed to have a comparison between the two architectures.

Embedding space of the encoding is visually analyzed with t-distributed stochastic neighbor embedding and the resulting visualization shows upsides and downsides of the encodings. Possible settings of encoding dimensionality are evaluated on the validation dataset and the results are compared according to the median accuracy and the variance of results. Finally, a self-supervised model is compared to a model trained with supervision. Datasets with different amount of annotated samples were used for training and validation accuracy of both models was compared. The self-supervised model performs similarly on dataset with hundreds of samples from each class but when the amount of samples drops to lower tens or even under 10 per class, the self-supervised model outperforms the supervised one by tens of percent on the validation accuracy.

Future work on the project was proposed in various directions. To increase the variability in sports poses, yoga poses are recommended together with possible sources of data and related work in this direction. Numerous other loss functions that can be used for self-supervision are presented together with their advantages. The obtained embeddings can not only be used for classification but also other computer vision challenges. These possibilities such as pose estimation or embedding vector operations are discussed. Lastly, the implemented work could also be used for other object classification, not only for sports poses. 
