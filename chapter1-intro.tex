\chapter{\label{chap:intro}Introduction}

Learning algorithms of many models used for recognition are nowadays relying on large datasets and also high computational power. Whereas the computational power of computers is still expected to increase, the creation of larger and more accurate annotated datasets is a task much harder. Labeling usually requires some sort of manual work that cannot be automated unless the required task was previously solved. This problem can be minimized by using unlabeled data to learn their representations and then needing only a~smaller amount of labeled samples to provide good classification results. This method is referred to as self-supervised learning.

The introduction to machine learning and convolutional neural networks is covered at~the~beginning of Chapter~\ref{chap:nn-in-cv}. Then, important features of convolutional neural networks that are used in this thesis are described. After that, current advances in sports pose recognition are presented. The following section introduces self-supervised learning and its advantages and disadvantages in deep learning. Finally, various loss functions for self-supervised learning are discussed.

This thesis discusses self-supervision achieved with a time-contrastive approach when one scene is filmed from multiple viewpoints and the difference in time and camera position is used as supervision that needs no labeling. The recorded video footage has to be normalized to the network's input, synchronized, and then motion detection is needed to produce individual training images. This whole process is specified in detail in Chapter~\ref{chap:dataset}. A~set of tools for semi-automatic preparation of such dataset is proposed. Finally, 2 different datasets with multiple sets of labels are presented. The datasets were recorded for this thesis and were prepared with the introduced tools.

The classifier model can be divided into 2 parts: the first one encodes the image into an~embedding and the second one recognizes the sports pose from the embedding. Chapter~\ref{chap:classification} describes both of these models and their specific parameters. Inputs of the first model are images with sports poses and outputs are their representations -- embeddings in a form of latent vectors. The first model is a convolutional neural network trained with triplet loss function, the second model is a simple neural network with densely connected layers, which classifies the embeddings into given categories.

Evaluation of the performance of all described models is provided in Chapter~\ref{chap:evaluation}. At~first, class distribution is visually analyzed with t-distributed stochastic neighbor embedding to better understand the embedding space. Afterward, the models are evaluated on validation data and their different settings are compared to each other. Finally, the overall performance of the self-supervised model on sports poses is measured and compared to the supervised model. All evaluations are discussed and their key takeaways are emphasized.
