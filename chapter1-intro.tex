\chapter{\label{chap:intro}Introduction}

Learning algorithms of many models used for recognition are nowadays relying on large datasets and also high computational power. Whereas computational power of computers is still expected to increase, creation of larger and more accurate annotated datasets is a task much harder. Labeling usually requires some sort of manual work that cannot be automated unless the required task was previously solved. This problem can be minimized with using unlabeled data to learn their representations and then needing only smaller amount of labeled samples to provide good classification results. This method is referred to as self-supervised learning.

The introduction to machine learning and convolutional neural networks is covered in the beginning of Chapter~\ref{chap:nn-in-cv}. Then, important features of convolutional neural networks that are used in this thesis are described. After that, current advances in sports pose recognition are presented. Following section introduces self-supervised learning and its advantages and disadvantages in deep learning. Finally, various loss functions for self-supervised learning are discussed.

Self-supervision can be achieved in different ways depending on the task that is being solved and on the available data. This thesis discusses time-contrastive approach when one scene is filmed from multiple viewpoints and the difference in time and camera position is used as a supervision that needs no labeling. After the footage is recorded and synchronized, it has to be adjusted for the network's input. High quality movement detection is necessary to determine differences between frames to produce images for the dataset. This whole process is specified in detail in Chapter~\ref{chap:dataset}.

The classifier model can be divided into 2 parts -- first one encodes the image into an embedding and the second one recognizes the sports pose from the embedding. Chapter~\ref{chap:classification} describes both of these models and their implementations in \texttt{Python 3} programming language with \texttt{TensorFlow 2} library. Inputs of the first model are images with sports poses and outputs are their representations -- embeddings in a form of latent vectors. The first model is a convolutional neural network trained with triplet loss function, the second model is a simple neural network with a densely connected layers, which classifies the embeddings into given categories.

Evaluation of performance of all described models is provided in Chapter~\ref{chap:evaluation}. At first, class distribution is visually analyzed with t-distributed stochastic neighbor embedding to better understand the embedding space. Afterwards, the models are evaluated on validation data and their different settings are compared to each other. Finally, the overall performance of self-supervised model on sports poses is measured and compared to supervised model. All evaluations are discussed and their key takeaways are emphasized.
