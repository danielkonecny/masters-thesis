\chapter{\label{chap:intro}Introduction}

Learning algorithms of many models used for recognition are nowadays relying on large datasets and also high computational power. Whereas computational power of computers is still expected to increase, creation of larger and more accurate datasets is a task much harder. Since most of the learning algorithms are supervised, training data has to be annotated. Labeling usually requires some sort of manual work that cannot be automated unless the required task was previously solved. The requirement of large annotated dataset can be minimized with using unlabeled data to learn their representations and then needing only smaller amount of labeled samples to provide good classification results. This method is referred to as self-supervised learning.

The introduction to Machine Learning and Convolutional Neural Networks is covered in the beginning of chapter \ref{chap:nn-in-cv}. These networks are widely used in image processing and especially in recognition tasks, which is also the main topic of this thesis. The current state-of-the-art models trained using supervised learning are introduced together with their main features. After that, current advances in sports pose recognition are presented. Following section introduces self-supervised learning and its advantages and disadvantages in deep learning. Finally, various loss functions for self-supervised learning are discussed.

Multiple approaches can be chosen to obtain and use the dataset. This thesis discusses the approach of filming one scene with multiple cameras and synchronizing the footage. The whole process is specified in chapter \ref{chap:dataset} from recording the scene and synchronizing the footage to creating the already usable dataset. Other approaches to creating a dataset suitable for this task and training procedure are also discussed.

Implementations of models that learn how to represent sports poses in latent space are described in chapter \ref{chap:representing}. \texttt{Python 3} programming language was used mostly together with \texttt{TensorFlow 2} library. Inputs of these models are images with sports poses, often structured into tuples, and outputs are their representations -- latent vectors. The first implemented model is a Convolutional Neural Network trained with triplet loss function. This model is then iteratively improved to provide better results.

Sport poses represented as vectors in a latent space have to be then classified to given categories. Possible solutions for this supervised learning tasks are provided in chapter \ref{chap:recognizing}. To better understand the data distribution, latent space is primarily analyzed with visual methods. Afterwards, multiple classifiers are introduced to solve this problem and their performances are analyzed. Finally, the overall performance in sports pose recognition is measured and compared to other models.