\chapter{\label{chap:recognizing}Recognition of Sports Poses from Latent Vectors}

The sports poses are encoded as $d$-dimensional latent vectors. At first, this chapter shows a couple of data visualization to better understand the data distribution. Then, simple models that solve the classification problem are proposed. These models do not require many annotated samples to have good performance. After that, more advanced models are proposed.

\blindtext

\section{Analyzing Latent Space with Visual Methods}

The latent space has well over 3 dimensions and therefore cannot be easily visualized.
\blindtext

\blindtext

\blindtext

\subsection{Visualization with Linear Discriminant Analysis}

Linear discriminant analysis (LDA) uses covariance matrices of the annotated data to transform them into a space with less dimensions that can be displayed.
\blindtext

\blindtext

\blindtext

\begin{figure*}[ht]\centering
  \centering
  \includegraphics[width=4in,height=2in]{figures/placeholder.pdf}
  \caption{Poses as different colored points in latent space transformed with LDA.}
  \label{LDA}
\end{figure*}

\blindtext

\subsection{Visualization with t-distributed Stochastic Neighbor Embedding}

t-distributed stochastic neighbor embedding (t-SNE) is a nonlinear dimensionality reduction method suited for displaying embedding vectors in two or three-dimensional space \cite{sne-NIPS2002_6150ccc6} \cite{tsne-vandermaaten08a}. 
\blindtext

\blindtext

\blindtext

\begin{figure*}[ht]\centering
  \centering
  \includegraphics[width=4in,height=2in]{figures/placeholder.pdf}
  \caption{Poses as different colored points in latent space transformed with t-SNE.}
  \label{tSNE}
\end{figure*}

\blindtext

\section{Simple Models for Sports Pose Classification}

Possible proposed models for the classification are k-nearest neighbors classifier and support vector machine.
\blindtext

\blindtext

\subsection{K-Nearest Neighbors Classifier}

K-nearest neighbors classifier is try to find the prevailing class of the nearest latent vectors to classify the new sample.
\blindtext

\blindtext

\blindtext

\subsection{Support-Vector Machine}

Support-vector machine (SVM) is trying to find a linear separator that maximizes the gap between two classes. It is possible to generalize SVMs to solve multi-class problems and non-linearities can be solved with non-linear projections.
\blindtext

\blindtext

\blindtext

\begin{figure*}[ht]\centering
  \centering
  \includegraphics[width=4in,height=2in]{figures/placeholder.pdf}
  \caption{Transformed latent space with transformed division line of SVM.}
  \label{SVM}
\end{figure*}

\blindtext

\subsection{Other simple models?}

\blindtext

\blindtext

\blindtext

\section{Advanced Models for Sports Pose Classification}

The sports pose latent vector classification can be solved with more advanced models, such as neural networks with dense layers.
\blindtext

\blindtext

\subsection{Neural Networks with Dense Layers}

\blindtext

\blindtext

\subsection{Other advanced models?}

\blindtext

\blindtext

\blindtext

\section{Comparison of Self-Supervised and Supervised-Trained Models}

\blindtext

\blindtext

\blindtext
